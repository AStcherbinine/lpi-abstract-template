% This is an LPSC Abstract template for LaTeX 2e that is based off of the
% LaTeX article document class.

% Copyright (C) 2005,2007 Ross A. Beyer
% Copyright (C) 2008 Ross A. Beyer and Moses P. Milazzo
% Copyright (C) 2018 Benoit Seignovert
% Copyright (C) 2024 Aurélien Stcherbinine

% This work is licensed under the Creative Commons
% Attribution-Noncommercial-Share Alike 4.0 License. To view a copy
% of this license, visit http://creativecommons.org/licenses/by-nc-sa/4.0/;
% or, (b) send a letter to Creative Commons, 171 2nd Street, Suite
% 300, San Francisco, California, 94105, USA.

\documentclass{lpsc_abstract} 

\title{Replace this sentence with the title of your abstract}

\author[1]{A. B. Author}
\author[2]{C. D. Author}

\affil[1]{Affiliation (include full mailing address and e-mail address if desired) for first author}
\affil[2]{Affiliation for second author (full mailing address and e-mail address)}

\hypersetup{
  pdfcreator={A. Author},
  pdfsubject={Conference name},
}

\begin{document}

\maketitle    % Nominal template, title, authors and affiliation on the same line
%\Maketitle   % Linebreak after title and authors list

\paragraph{Introduction:}
Replace these instructions with the text of your abstract.
Do NOT delete the section break above.
The text will appear in two columns that are each \SI{\sim 3}{inches} wide.
(Please make sure your paper size is set to U.S. letter, \SIrange[range-phrase=$\times$]{8.5}{11}{"}, before submitting your abstract.)
Page margins are set to be one inch on all sides.
If you are including tables or figures, they MUST be imported into this file.

This template is designed to accommodate abstracts for the majority of USRA meetings; it is your responsibility to be aware of the guidelines (length restrictions, file size, etc.) for a particular meeting. These will be detailed in the Call for Abstracts section of the meeting announcement.

The text will automatically wrap to a second page if necessary. The running head on the second page of this template has been eliminated intentionally.

\paragraph{Digital Formats:}
Any image file format that can be imported into this file will be acceptable for publication; to avoid technical problems, we suggest using TIFF (.tif) or GIF (.gif) files for photographs, and encapsulated PostScript (.eps) or Windows metafiles (.wmf) for line drawings.
We ask that you use smaller-format files whenever possible (for example, don’t use a 1-MB TIFF file if a 250-K GIF file provides acceptable resolution).

\paragraph{Heading Styles:}
The section heads in this template use the correct style (upper and lower case, bold, followed by a colon). The format for second-level heads is show below:

\subparagraph{Sample of a level-two head.}
Automatic paragraph indents are imbedded to appear every time you use a hard return. If you’re using the spell checker and it doesn’t seem to be working, check the “Language” option under the “Tools” to make sure that “No Proofing” isn’t selected as the default. You should also check that the default font is Times New Roman and not New York (this conversion takes place on some Macintosh systems).

\paragraph{Use of Meteorite Names:}
All meteorites that are cited in abstracts must have official names approved by the Nomenclature Committee. Authors can use the \href{http://www.lpi.usra.edu/meteor/}{Meteoritical Bulletin Database} to check the status of a meteorite name. New meteorites without approved names may be rejected. The instructions for submitting new meteorite names can be found on the \href{http://meteoriticalsociety.org/?page_id=63}{Meteoritical Society website}.

The full names of meteorites should be used in titles, subheadings, and at first mention in the text. Abbreviations, including those published in the Antarctic Meteorite Newsletter and the Meteoritical Bulletin, may be used in tables and elsewhere. Note that in the abbreviated form, there should be a space between the place name and the number.
In addition, Antarctic meteorites recovered prior to 1981 may have an A after the blank space. Please see the \href{http://meteoriticalsociety.org/?page_id=61}{list of standard abbreviations and examples} of their proper usage.

\paragraph{References:}
Use the brief numbered style common in many abstracts, e.g., \cite{article1}, \cite{article2}, etc. References should then appear in numerical order in the reference list, and should use the following abbreviated style:
\nocite{article3,article4}

\bibliography{bibliography}

\paragraph{Additional Information:}
If you have any questions or need additional information
regarding the preparation of your abstract, call the LPI at 281-486-2142
or -2188 (or send an e-mail message to \href{mailto:publish@hou.usra.edu}{publish@hou.usra.edu}).

\paragraph{Please DO NOT Submit Duplicates of Your Abstract;} should you find it necessary to replace or repair your abstract PRIOR TO the submission dead-line, return to the \href{https://www.hou.usra.edu/meeting_portal/abstract_submission/}{abstract submission portion} of the meeting portal and click on the “Update” link that ap-pears next to the title of the abstract you submitted.
 
% \paragraph{Figures:}
% Uncomment this paragraph to enable figure input

% \begin{figure}[!ht]
%     \label{color_scales}
%     \centering
%     \includegraphics[width=\columnwidth]{figure.png}
%     \caption{This is a figure caption.}
% \end{figure}

\end{document}
